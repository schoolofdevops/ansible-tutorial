\chapter{Variables and Templates}

In this  tutorial, we  are going to make the roles that we created earlier dynamically by adding templates and defining variables.

\section{Variables}

Variables are of two types

\begin{itemize}
\item Automatic Variables/ Facts
\item User Defined Variables
\end{itemize}

Lets try to discover information about our systems by using facts.

\section{Finding Facts About Systems}

\begin{itemize}
\item Run the following command to see to facts of db servers
\end{itemize}

%= lang:text
\begin{code}
ansible db -m setup
\end{code}

[Output]\newline
%= lang:text
\begin{code}
192.168.61.11 | SUCCESS => {
        "ansible_facts": {
            "ansible_all_ipv4_addresses": [
                "10.0.2.15",
                "192.168.61.11"
            ],
            "ansible_all_ipv6_addresses": [
                "fe80::a00:27ff:fe30:3251",
                "fe80::a00:27ff:fe8e:83e0"
            ],
            "ansible_architecture": "x86_64",
            "ansible_bios_date": "12/01/2006",
            "ansible_bios_version": "VirtualBox",
            "ansible_cmdline": {
                "KEYBOARDTYPE": "pc",
                "KEYTABLE": "us",
                "LANG": "en_GB.UTF-8",
                "SYSFONT": "latarcyrheb-sun16",
                "quiet": true,
                "rd_LVM_LV": "VolGroup/lv_root",
                "rd_NO_DM": true,
                "rd_NO_LUKS": true,
                "rd_NO_MD": true,
                "rhgb": true,
                "ro": true,
                "root": "/dev/mapper/VolGroup-lv_root"
            },
            "ansible_date_time": {
                "date": "2016-09-05",
                "day": "05",
                "epoch": "1473104372",
                "hour": "20",
                "iso8601": "2016-09-05T19:39:32Z",
                "iso8601_basic": "20160905T203932677277",
                "iso8601_basic_short": "20160905T203932",
                "iso8601_micro": "2016-09-05T19:39:32.677365Z",
                "minute": "39",
                "month": "09",
                "second": "32",
                "time": "20:39:32",
                "tz": "BST",
                "tz_offset": "+0100",
                "weekday": "Monday",
                "weekday_number": "1",
                "weeknumber": "36",
                "year": "2016"
            }

\end{code}

\section{Filtering facts}

\begin{itemize}
\item Use filter attribute to extract specific data
\end{itemize}

%= lang:text
\begin{code}
ansible db -m setup -a "filter=ansible_distribution"
\end{code}

[Output]\newline
%= lang:text
\begin{code}
192.168.61.11 | SUCCESS => {
  "ansible_facts": {
      "ansible_distribution": "CentOS"
  },
  "changed": false
}  
\end{code}

\section{Registering  Variables}

Lets create a playbook to run a shell command, register the result and display the value of registered variable.

Create \textbf{register.yml} in chap6 directory

%= lang:text
\begin{code}
---
  - name: register variable example
    hosts: local
    tasks:
      - name: run a shell command and register result
        shell: "/sbin/ifconfig eth1"
        register: result

      - name: print registered variable
        debug: var=result
\end{code}

Execute the playbook to display information about the registered variable

%= lang:text
\begin{code}
ansible-playbook  register.yml
\end{code}

\section{Creating Templates for Apache}

\begin{itemize}
\item Create template for apache configuration
\item This template will change \textbf{port number}, \textbf{document root} and \textbf{index.html} for  apache server
\item Copy \textbf{httpd.conf} file from \textbf{roles/apache/files/} to \textbf{roles/apache/templates}
\end{itemize}

%= lang:text
\begin{code}
cp roles/apache/files/httpd.conf roles/apache/templates/httpd.conf.j2
\end{code}

\begin{itemize}
\item Change your working directory to templates
\end{itemize}

%= lang:text
\begin{code}
cd roles/apache/templates
\end{code}

\begin{itemize}
\item Change values of following  parameters by using template variables in \textbf{httpd.conf.j2}


\begin{itemize}
\item Listen
\item DocumentRoot
\item DirectoryIndex
\end{itemize}
\end{itemize}

Following code depicts only the parameters changed. Rest of the configurations in \emph{httpd.conf.j2} remain as is

%= lang:text
\begin{code}
Listen {{ apache_port }}
DocumentRoot "{{ custom_root }}"
DirectoryIndex {{ apache_index }} index.html.var
\end{code}

\begin{itemize}
\item Create a template for index.html as well
\end{itemize}

%= lang:text
\begin{code}
cp roles/apache/files/index.html roles/apache/templates/index.html.j2
\end{code}

\begin{itemize}
\item Add the following contents to index.html.j2
\end{itemize}

%= lang:text
\begin{code}
<html>
<body>
  <h1>  Welcome to Ansible training! </h1>

  <h2> SYSTEM INFO </h2>
  <h4>  ========================= </h4>
  <h3> Operating System : {{ ansible_distribution }} </h3>
  <h3> IP Address : {{ ansible_eth1['ipv4']['address'] }} </h3>

  <h2>  My Favourites </h2>
  <h4>  ========================= </h4>

  <h3> color     : {{ fav['color'] }} </h3>
  <h3> fruit     : {{ fav['fruit']   }} </h3>
  <h3> car       : {{ fav['car']   }} </h3>
  <h3> laptop    : {{ fav['laptop']   }} </h3>
  <h4>  ========================= </h4>
</body>
</html>
\end{code}

\section{Defining Default Variables}

\begin{itemize}
\item Define values of the variables used in the templates above.  The default values are defined in \emph{roles/apache/defaults/main.yml} . Lets edit that file and add the following,
\end{itemize}

%= lang:text
\begin{code}
apache_port: 80
custom_root: /var/www/html
apache_index: index.html
fav:
  color: white
  car: fiat
  laptop: dell
  fruit: apple
\end{code}

\section{Updating Tasks to use Templates}

\begin{itemize}
\item Since we are now using template instead of static file, we need to edit \emph{roles/apache/tasks/config.yml} file and use template module
\item Replace \textbf{copy} module with \textbf{template} modules as follows,
\end{itemize}

%= lang:text
\begin{code}
---
- name: Creating configuration from templates...
  template: src=httpd.conf.j2
            dest=/etc/httpd.conf
            owner=root group=root mode=0644
  notify: Restart apache service
- name: Copying index.html file...
  template: src=index.html.j2
        dest=/var/www/html/index.html
        mode=0777
\end{code}

\begin{itemize}
\item Delete httpd.conf and index.html in files directory
\end{itemize}

%= lang:text
\begin{code}
rm roles/apache/files/httpd.conf
rm roles/apache/files/index.html
\end{code}

\subsection{Validating}

\begin{itemize}
\item Let's test this template in action
\end{itemize}

%= lang:text
\begin{code}
ansible-playbook app.yml
\end{code}

[Output]\newline
%= lang:text
\begin{code}
PLAY [Playbook to configure App Servers] ***************************************

TASK [setup] *******************************************************************
ok: [192.168.61.13]
ok: [192.168.61.12]

TASK [base : create admin user] ************************************************
ok: [192.168.61.13]
ok: [192.168.61.12]

TASK [base : remove dojo] ******************************************************
ok: [192.168.61.12]
ok: [192.168.61.13]

TASK [base : install tree] *****************************************************
ok: [192.168.61.12]
ok: [192.168.61.13]

TASK [base : install ntp] ******************************************************
ok: [192.168.61.13]
ok: [192.168.61.12]

TASK [base : start ntp service] ************************************************
ok: [192.168.61.13]
ok: [192.168.61.12]

TASK [apache : Installing Apache...] *******************************************
ok: [192.168.61.13]
ok: [192.168.61.12]

TASK [apache : Starting Apache...] *********************************************
ok: [192.168.61.13]
ok: [192.168.61.12]

TASK [apache : Creating configuration templates...] ****************************
changed: [192.168.61.12]
changed: [192.168.61.13]

TASK [apache : Copying index.html file...] *************************************
changed: [192.168.61.12]
changed: [192.168.61.13]

RUNNING HANDLER [apache : Restart apache service] ******************************
changed: [192.168.61.12]
changed: [192.168.61.13]

PLAY RECAP *********************************************************************
192.168.61.12              : ok=11   changed=3    unreachable=0    failed=0
192.168.61.13              : ok=11   changed=3    unreachable=0    failed=0
\end{code}

\section{Playing with Variable Precedence Rules}

Lets define the variables from couple of other places, to learn about the Precedence rules. We will create,

\begin{itemize}
\item group\_vars
\item playbook vars
\end{itemize}

Since we are going to define the variables using multi level hashes, lets define the way hashes behave when defined from multiple places.

Update chapter6/ansible.cfg and add the following,

%= lang:text
\begin{code}
hash_behaviour=merge
\end{code}

Lets create group\_vars and create a group \textbf{all} to define vars common to all.

%= lang:text
\begin{code}
cd /vagrant/code/chap6
mkdir group_vars
cd group_vars
touch all.yml
\end{code}

Edit \textbf{group\_vars/all} file and add the following contents,

%= lang:text
\begin{code}
---
  fav:
    color: blue
    fruit: peach

\end{code}

Lets also add vars to playbook. Edit app.yml and add vars as below,

%= lang:text
\begin{code}
---
  - name: Playbook to configure App Servers
    hosts: app
    become: true
    vars:
      fav:
        fruit: mango
    roles:
    - apache
\end{code}

Execute the playbook and check the output

%= lang:text
\begin{code}
ansible-playbook app.yml
\end{code}

If you view the content of the html file generated, you would notice the following,

%= lang:text
\begin{code}
<h3> color     : blue </h3>
<h3> fruit     : mango </h3>
<h3> car       : fiat </h3>
<h3> laptop    : dell </h3>
\end{code}

\begin{itemize}
\item value of color comes from group\_vars/all.yml
\item value of fruit comes from playbook vars
\item value of car and laptop comes from role defaults
\end{itemize}

\section{Exercises}

\begin{itemize}
\item Create host specific variables in host\_vars/HOSTNAME for one of the app servers, and define some variables values specific to the host. See the output after applying playbook on this node.  
\item Generate MySQL Configurations dynamically using templates and modules.  

\begin{itemize}
\item Create a template for my.cnf.  Name it as roles/mysql/templates/my.cnf.j2  
\item Replace parameter values with templates variables  
\item Define variables in role defaults.
\end{itemize}
\end{itemize}