\chapter{Learning to Write Playbooks}

In this tutorial we are going to create a simple playbook to add system users, install and start ntp service and some basic utilities.

\subsection{Creating our first playbook}

\begin{itemize}
\item Change working directory to /vagrant/code/chap4\newline
%= lang:text
\begin{code}
     cd /vagrant/code/chap4
```  
  * Create playbook.yml and add the content below

\end{code}
---
\item name: Base Configurations for ALL hosts
hosts: all
become: true
tasks:


\begin{itemize}
\item name: create admin user
user: name=admin state=present uid=5001
\item name: remove dojo
user: name=dojo  state=absent
\item name: install tree
yum:  name=tree  state=present
\item name: install ntp
yum:  name=ntp   state=present
\item name: start ntp service
service: name=ntpd state=started enabled=yes
```
\end{itemize}
\end{itemize}

\subsection{Running the  playbook}

To run the playbook, we are going to execute \textbf{ansible-playbook} command. Lets first examine the options that this command supports.

%= lang:text
\begin{code}
ansible-playbook --help

\end{code}

%= lang:text
\begin{code}
[output]

Usage: ansible-playbook playbook.yml

Options:
  --ask-become-pass     ask for privilege escalation password
  -k, --ask-pass        ask for connection password
  --ask-su-pass         ask for su password (deprecated, use become)
  -K, --ask-sudo-pass   ask for sudo password (deprecated, use become)
  --ask-vault-pass      ask for vault password
  -b, --become          run operations with become (nopasswd implied)
  --become-method=BECOME_METHOD
                        privilege escalation method to use (default=sudo),
                        valid choices: [ sudo | su | pbrun | pfexec | runas |
                        doas ]

.......
\end{code}

To run the playbook, we could call YAML file as an argument. Since we have already defined the inventory and configurations, additional options are not necessary at this time.

%= lang:text
\begin{code}
ansible-playbook playbook.yml
\end{code}

%= lang:text
\begin{code}
[output]

PLAY [Base Configurations for ALL hosts] ***************************************

TASK [setup] *******************************************************************
ok: [192.168.61.14]
ok: [192.168.61.11]
ok: [localhost]
ok: [192.168.61.12]
ok: [192.168.61.13]

TASK [create admin user] *******************************************************
changed: [192.168.61.13]
changed: [192.168.61.12]
changed: [localhost]
changed: [192.168.61.11]
changed: [192.168.61.14]

TASK [remove dojo] *************************************************************
changed: [192.168.61.14]
changed: [localhost]
changed: [192.168.61.12]
changed: [192.168.61.11]
changed: [192.168.61.13]

TASK [install tree] ************************************************************
ok: [localhost]
ok: [192.168.61.13]
ok: [192.168.61.12]
ok: [192.168.61.14]
ok: [192.168.61.11]

TASK [install ntp] *************************************************************
changed: [192.168.61.12]
changed: [192.168.61.13]
changed: [192.168.61.11]
changed: [localhost]
changed: [192.168.61.14]

TASK [start ntp service] *******************************************************
changed: [192.168.61.11]
changed: [localhost]
changed: [192.168.61.13]
changed: [192.168.61.12]
changed: [192.168.61.14]
\end{code}

\subsection{Adding second play in the playbook}

Lets add a second play specific to app servers. Add the following block of code in playbook.yml file and save   

%= lang:text
\begin{code}
- name: App Server Configurations
  hosts: app
  become: true
  tasks:
    - name: create app user
      user: name=app state=present uid=5003

    - name: install git
      yum:  name=git  state=present

```  

Run the playbook again...  

\end{code}
ansible-playbook playbook.yml
%= lang:text
\begin{code}

\end{code}

PLAY [Base Configurations for ALL hosts] \textbf{**}\textbf{**}\textbf{**}\textbf{**}\textbf{**}\textbf{**}***

TASK [setup] \textbf{**}\textbf{**}\textbf{**}\textbf{**}\textbf{**}\textbf{**}\textbf{**}\textbf{**}\textbf{**}\textbf{**}\textbf{**}*
ok: [localhost]
ok: [192.168.61.14]
ok: [192.168.61.13]
ok: [192.168.61.12]
ok: [192.168.61.11]

TASK [create admin user] \textbf{**}\textbf{**}\textbf{**}\textbf{**}\textbf{**}\textbf{**}\textbf{**}\textbf{**}\textbf{**}*
ok: [192.168.61.14]
ok: [localhost]
ok: [192.168.61.13]
ok: [192.168.61.12]
ok: [192.168.61.11]

TASK [remove dojo] \textbf{**}\textbf{**}\textbf{**}\textbf{**}\textbf{**}\textbf{**}\textbf{**}\textbf{**}\textbf{**}\textbf{**}*
ok: [192.168.61.14]
ok: [192.168.61.12]
ok: [192.168.61.11]
ok: [localhost]
ok: [192.168.61.13]

TASK [install tree] \textbf{**}\textbf{**}\textbf{**}\textbf{**}\textbf{**}\textbf{**}\textbf{**}\textbf{**}\textbf{**}\textbf{**}
ok: [192.168.61.11]
ok: [192.168.61.12]
ok: [localhost]
ok: [192.168.61.13]
ok: [192.168.61.14]

TASK [install ntp] \textbf{**}\textbf{**}\textbf{**}\textbf{**}\textbf{**}\textbf{**}\textbf{**}\textbf{**}\textbf{**}\textbf{**}*
ok: [192.168.61.12]
ok: [192.168.61.13]
ok: [192.168.61.14]
ok: [192.168.61.11]
ok: [localhost]

TASK [start ntp service] \textbf{**}\textbf{**}\textbf{**}\textbf{**}\textbf{**}\textbf{**}\textbf{**}\textbf{**}\textbf{**}*
ok: [192.168.61.13]
ok: [192.168.61.14]
ok: [localhost]
ok: [192.168.61.11]
ok: [192.168.61.12]

PLAY [App Server Configurations] \textbf{**}\textbf{**}\textbf{**}\textbf{**}\textbf{**}\textbf{**}\textbf{**}*****

TASK [setup] \textbf{**}\textbf{**}\textbf{**}\textbf{**}\textbf{**}\textbf{**}\textbf{**}\textbf{**}\textbf{**}\textbf{**}\textbf{**}*
ok: [192.168.61.13]
ok: [192.168.61.12]

TASK [create app user] \textbf{**}\textbf{**}\textbf{**}\textbf{**}\textbf{**}\textbf{**}\textbf{**}\textbf{**}\textbf{**}***
changed: [192.168.61.12]
changed: [192.168.61.13]

TASK [install git] \textbf{**}\textbf{**}\textbf{**}\textbf{**}\textbf{**}\textbf{**}\textbf{**}\textbf{**}\textbf{**}\textbf{**}*
ok: [192.168.61.13]
ok: [192.168.61.12]

PLAY RECAP \textbf{**}\textbf{**}\textbf{**}\textbf{**}\textbf{**}\textbf{**}\textbf{**}\textbf{**}\textbf{**}\textbf{**}\textbf{**}***
192.168.61.11              : ok=6    changed=0    unreachable=0    failed=0
192.168.61.12              : ok=9    changed=1    unreachable=0    failed=0
192.168.61.13              : ok=9    changed=1    unreachable=0    failed=0
192.168.61.14              : ok=6    changed=0    unreachable=0    failed=0
localhost                  : ok=6    changed=0    unreachable=0    failed=0
%= lang:text
\begin{code}

### Limiting the execution to a particular group  

Now run the following command to restrict the playbook execution to *app servers*  

\end{code}
ansible-playbook playbook.yml --limit app

%= lang:text
\begin{code}

This will give us the following output, plays will be executed only on app servers...  


\end{code}
PLAY [Base Configurations for ALL hosts] \textbf{**}\textbf{**}\textbf{**}\textbf{**}\textbf{**}\textbf{**}***

TASK [setup] \textbf{**}\textbf{**}\textbf{**}\textbf{**}\textbf{**}\textbf{**}\textbf{**}\textbf{**}\textbf{**}\textbf{**}\textbf{**}*
ok: [192.168.61.13]
ok: [192.168.61.12]

TASK [create admin user] \textbf{**}\textbf{**}\textbf{**}\textbf{**}\textbf{**}\textbf{**}\textbf{**}\textbf{**}\textbf{**}*
ok: [192.168.61.13]
ok: [192.168.61.12]

TASK [remove dojo] \textbf{**}\textbf{**}\textbf{**}\textbf{**}\textbf{**}\textbf{**}\textbf{**}\textbf{**}\textbf{**}\textbf{**}*
ok: [192.168.61.12]
ok: [192.168.61.13]

TASK [install tree] \textbf{**}\textbf{**}\textbf{**}\textbf{**}\textbf{**}\textbf{**}\textbf{**}\textbf{**}\textbf{**}\textbf{**}
ok: [192.168.61.13]
ok: [192.168.61.12]

TASK [install ntp] \textbf{**}\textbf{**}\textbf{**}\textbf{**}\textbf{**}\textbf{**}\textbf{**}\textbf{**}\textbf{**}\textbf{**}*
ok: [192.168.61.12]
ok: [192.168.61.13]

TASK [start ntp service] \textbf{**}\textbf{**}\textbf{**}\textbf{**}\textbf{**}\textbf{**}\textbf{**}\textbf{**}\textbf{**}*
ok: [192.168.61.12]
ok: [192.168.61.13]

PLAY [App Server Configurations] \textbf{**}\textbf{**}\textbf{**}\textbf{**}\textbf{**}\textbf{**}\textbf{**}*****

TASK [setup] \textbf{**}\textbf{**}\textbf{**}\textbf{**}\textbf{**}\textbf{**}\textbf{**}\textbf{**}\textbf{**}\textbf{**}\textbf{**}*
ok: [192.168.61.13]
ok: [192.168.61.12]

TASK [create app user] \textbf{**}\textbf{**}\textbf{**}\textbf{**}\textbf{**}\textbf{**}\textbf{**}\textbf{**}\textbf{**}***
ok: [192.168.61.13]
ok: [192.168.61.12]

TASK [install git] \textbf{**}\textbf{**}\textbf{**}\textbf{**}\textbf{**}\textbf{**}\textbf{**}\textbf{**}\textbf{**}\textbf{**}*
ok: [192.168.61.12]
ok: [192.168.61.13]

PLAY RECAP \textbf{**}\textbf{**}\textbf{**}\textbf{**}\textbf{**}\textbf{**}\textbf{**}\textbf{**}\textbf{**}\textbf{**}\textbf{**}***
192.168.61.12              : ok=9    changed=0    unreachable=0    failed=0
192.168.61.13              : ok=9    changed=0    unreachable=0    failed=0

%= lang:text
\begin{code}


## Exercise:
Create a Playbook with the following specifications,
  * It should apply only on local host (ansible host)
  * Should use become method
  * Should create a **user** called webadmin with shell as "/bin/sh"
  * Should install and start **nginx** service
  * Should **deploy** a sample html app into the default web root directory of nginx using ansible's **git** module.
    * Source repo:  https://github.com/schoolofdevops/html-sample-app
    * Deploy Path : /usr/share/nginx/html/app
    * The user to deploy the app would be nginx


#####  TODO for Course Creator:
   - Fail the task (w.g. service name = ntp).
   - It will create a retry file
   Use that to feed into to ansible-playbook with --limit option
   e.g.
   ```
    ansible-playbook playbook.yml --limit @/tmp/playbook.rerty
   ```
\end{code}